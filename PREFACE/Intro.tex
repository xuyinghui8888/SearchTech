
\markboth{内容简介}{内容简介} \vspace*{0.0cm}
\thispagestyle{empty}
\vspace*{2.2cm}
\centerline{\zihao{2}\hei{\color{darkblue}{内~容~简~介}}}\vspace{2cm}

各类主要的互联网服务,包括搜索、广告、推荐、等等,它们的一个典型共同特征,
就是利用不断增强的计算处理能力和日益丰富的资源种类,对百万乃至上百亿量级以上的超大规模数据进行分析和挖掘,
数据维度包罗万象,比如网页内容、用户行为、时间序列、等等,去充分理解消费者需求,
定位供给端的品类和质量,建立一个良性的买家和卖家的公平交易平台;

淘宝搜索作为平台的一个重要联系买家和卖家的产品形态, 
由于其以下的特有属性,使其成为大数据智能化应用的最佳场景;
\begin{description}
	\item 海量消费者与平台的互动行为
	\item 买家和卖家的公平交易平台
	\item 海量商家在平台进行的商业活动行为
\end{description} 

本书将围绕淘宝搜索智能化体系的演进历程进行系统化阐述,如何依托于工程架构体系的逐步完善,
逐步实现从简单人工运营加简单算法规则的时代,发展成为阿里电商平台辅助消费者与商品(卖家)
的互动更加趣味化和效率化的智能中枢,不仅仅可以从海量用户行为数据中寻找行为规律,结构化行
为序列,并从规律中预测结果,更重要的是给出有效的流量中心化和去中心化的投放决策,从而
实现消费者,卖家,平台三者社会福利的最大化。淘宝的搜所和推荐发展到今天,正在从智能的
依靠机器学习能力解决业务问题,向更高效的从不确定性中探索目标的学习+决策的能力进化。











