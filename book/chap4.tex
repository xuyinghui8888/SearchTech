
\chapter{个性化搜索背后的核心技术}
\thispagestyle{empty}

\setlength{\fboxrule}{0pt}\setlength{\fboxsep}{0cm}
\noindent\shadowbox{
\begin{tcolorbox}[arc=0mm,colback=lightblue,colframe=darkblue,title=学习目标与要求]
%\kai\textcolor{darkblue}{1.~~强化学习.} \\ 

\end{tcolorbox}}
\setlength{\fboxrule}{1pt}\setlength{\fboxsep}{4pt} 

进入正题之前首先谈一谈个性化方向后续需要关注的几个点,搜索个性化时至今日,已经成为
互联网网站的技术标配,虽然业界取得了一些成绩,但挑战仍然存在;

首先来看看为什么要作个性化,搜索中引入个性化的目的是什么
每天有 近30000个查询“连衣裙”的消费者,query + “user context” 的查询逻辑能够实现不同的消费群体看到不同商品投放结果,实现平台上人-货匹配在搜索流量上的个性化细分,比如说,“肥胖”的女性查询结果里面更多的展现宽松风格的商品,而消费能力高的消费者更多的展现高品位的大牌商品,从而达到流量投放效率的最大化;总而言之,目的是两个 : 
a). 提升流量匹配效率:具体表现在购物路径上的效果指标;
b). 改善宽泛query下得流量集中性,提升宽泛query下不同人群看到的展示商品不同,而带来成交商品和点击商品的丰富性;
总而言之:对于广大消费者,由于个性化能够细分搜索意图,拟合个体偏好,有助于更快捷找到需求;

弄清楚了搜索个性化的目的,下面想来澄清几个问题:

1. 个性化并不是定制化

Customization:实现的境界是 You are what you say you are? 或者说,平台按照系统理解的用户profile,并按照某种特定个性化规则去投放,即是,实现 you are what the system thinks you are. 对于针对用户的查询结果的信息展示是局限在explicit 的“feature”层面,比方说,按照购买力匹配规则,按照品牌偏好规则等等;定制化的好处的确能够在某种程度上带来强的个性化体验,但是带来的伤害也是不言而喻;
而Personalization是:根据用户行为所挖掘的偏好信息来进行展示商品的投放,即是,实现 you are what you click on and what you buy;对于针对用户的查询结果的商品展示是基于“内容和数据”层面;不去刻意的假设用户的行为是由于某个特定维度(人口统计类特征,偏好类特征,人群特征)造成的,消费者点击或成交行为的发生,是所有个性化信息的综合表现;为什么我在这里先提出这个问题,因为经常听到的很多关于,目前线上个性化效果不尽人意的反馈,在这里,也不去刻意回避我们自己的问题,个性化数据,模型的覆盖率,准确性和时效性等都需要进一步的优化和改进;然而,对于那些为了增强所谓的个性化体验而实行的规则式匹配逻辑,都是极其不科学的做法,对于消费者而言,他们需要的是找到一个符合他/她需求的商品,而个性化体验强弱与否并非是最终的目的,我相信的是,消费者不会因为我们预测到他的性别,购买力,偏好的品牌就做出点击或购买的决策,个性化是我们系统实现高效的【人-货】匹配效率的手段,并非是消费者的购物诉求;在这里也请从事个性化方面的运营,产品,甚至算法同学能时刻理解这点;

2. 不要陷入活跃/资深用户的悖论

正常的用户无论其活跃与否,都不会愿意浪费时间去填写所谓的友好的交互式表单来帮助系统去理解他们,从而得到更好的个性化体验;他们关注的是展示商品整体是否满足他们的需求,而不会去刻意的由于商品的某个维度匹配了他/她得某个偏好而做出最终的选择;这里列举一个曾经的产品设计,在搜索结果页,给出用户可以定制的个性化偏好交互界面,希望消费者能告诉我们他们的个性化profile,出发点是好的,结局大家懂的;

3. 个性化explore的重要性

随着个性化元素在搜索全链路的渗透,从query的个性化标注,海选的个性化召回,精排中的个性化排序因子,以及个性化rerank,个性化展示,使得最终呈现给用户的内容取决于系统底层根据用户历史行为所挖掘的个性化特征,人口统计学维度,兴趣点偏好维度,session级别实时特征,过度的”user specific historical behaviour driven“的个性化投放,会使得用户逐渐丧失对展示结果的新鲜感,并且视野变得越来越狭窄, 进而使得底层的用户数据模型丧失自我修复和自我扩展能力;因此一个完整的个性化体系,必须考虑explore机制的设计环节;

4. 个性化评估的方法论

要想推动个性化效果的正向迭代,首先需要建立起合理的效果评估体系;然而这仍然是一个很大的问题,学术界流行的准确率,召回率,F1值,AUC,RMSE,等都有很大的局限性,这一层面的评估,只能保证数据模型的正确性;而在实际工作中,这些指标上的不一定能保证线上效果的收益;因此我们需要第二层次的评估手段,来看个性化算法效果。实现个性化的投放效果,是系统层面的主动而为,而且并没有去引导消费者端在一次搜索看到展示结果后,做出选择。 因此在消费者不知情情况下,消费者的行为反馈可以用来作为个性化效果评估的一个手段。对于已经上线的个性化特征,需要部署相应的统计分析模块,在ABtest机制下,监控各个特征的覆盖率,以及覆盖流量下的点击率,转化率等;虽然无法直接统计到这些特征对于点击和转化带来的精确影响,但是通过追踪高权重user的体验 - 点击率,2跳率,转化率等,能够了解这些特征的影响趋势,及早发现问题;这里特别强调下,采用高权重user的行为数据来分析的原因是,高权重用户意味着是活跃用户,意味着行为丰富,而这类用户的个性化特征的表现会更有代表性;最后,我也来谈谈对于针对个性化效果的社会化评测的意见和想法,便于理解,就以用户购买力为例来讲讲,为了更好来把握该维度数据的有效性,经常利用的手段是社会化评测来给定一些查询下,看看展示结果里面展示商品的价格是否符合评测者的价格偏好,从表象上看,似乎没有问题,然而,这里面确有一个本质上的问题,我们限定了这些参与评测人得判断角度,只关注商品价格,并给出满意与否的结论,而在实际购物场景下,用户对于商品满意与否接受与否,并不是只限定在价格本身,因此这样的评测还是有一定的局限性;我个人的观点,还是去真实的模拟线上的判断环境,不去刻意要求消费者去关注某个固定的维度,只是给出site by site的结果,让用户判断哪边展示的商品更符合他的口味,当然,这不同site的展示结果的差异,背后只是某个维度的个性化带来的影响,这样去评测,才更加客观;总结一句话,就是众包评测的关键是,希望参与者能做出客观的反馈,不应该做任何主观性的引导;

5. 个性化体系对于系统和框架的影响

在搜索场景下实现个性化的效果,就需要去建模分析 【query-user-商品】三元组构成下得海量数据分析,数据是极端稀疏的,算法时间和空间的复杂性,都对于体统能很好的支持分布和并行的数据分析和建模能力提出了很高的要求;另外,用户偏好的时效性,也需要我们能够实现增量,实时计算能力,个性化的实施,使得传统引擎依赖的性能优化利器,cache机制无法施展手脚,因此对于引擎的创新性改造也提出了更高的要求;另外,个性化数据的挖掘都是存在不确定性的,如何来设计一套能够保证误差不会累积的算法体系,也就是说需要建立一套数据自我修复的实时反馈体系,来保证由消费者端实时获取的客观反馈数据参与到个性化投放环节来保证模型的自我修复能力快于数据误差的传播速度;这样才能保证数据产生的价值形成良性的循环,构成大数据生态体系;



个性化是一种解决“长尾需求”的方式,“长尾理论”说的是用户需求集中度越来越低,用户和用户之间不一样,
我们如何来区分这种不一样? 个性化搜索就是融合推荐元素,以实现:用户个体需求主导的 “pull” 式搜索加
平台以数据驱动的方式对用户进行“push”式相关信息推送;


综述性的东西,@三桐,@公达

\section{匹配学习}
	
\subsection{一阶人货匹配模型} 
	@公达 u2i,u2s,u2b 

\subsection{高阶人货匹配模型} 
	@公达 u2i2i,u2s2i,u2b2i 

\subsection{深度匹配模型} 

\subsection{序列匹配模型} 

前面三个章节我们递进的描述了用户与单个商品之间的匹配方式和模型。
然而,上述方法均假设用户的购物行为之间是独立的——并不存在依赖、相关或序列关系。

举例来说,一个用户$U_1$2天内依次购买了以下商品:烤箱、面粉、奶油;另一个用户$U_2$半年内依次购买了孕妇衣、尿布和奶瓶。我们先考虑用户$U_1$,我们可以从他购买了烤箱和面粉2种商品推断他很可能想要做蛋糕(而这从每个单一买的商品都是很难推断的),因此也许需要奶油;再考虑用户$U_2$,我们可以从她依次购买了孕妇衣和尿布推断她很可能怀孕过并且已经生了小宝宝(从某一件来推荐会比较勉强),因此马上会需要奶瓶等婴儿用品。

从上面例子我们可以看出,用户的购物行为之间往往是存在高阶依赖关系的,即仅用户购买了一个商品集合$\{A, B, C\}$后,才会购买商品$D$;同时,用户的购物行为也会存在序列关系,即用户购买C,仅会在他依次购买了商品A和B之后。在这2种关系下,我们前3节使用的模型会很难捕捉这类规律。因此我们需要一种模型,能整体的考虑用户的行为历史(而不是将其行为拆分成一个一个的单独分析),进而推断他接下来的需求。

下面我们会首先介绍几种经典的序列模型以及带有记忆功能的模型,然后会详细介绍在淘宝搜索中,我们怎样使用这类模型做到用户与商品之间的序列匹配。

在机器学习的任务环境中,我们有大量的场景都是需要做一个序列预测和带有记忆的推断的。例如在query自动补全的场景下,我们需要根据用户输入的文字或者词序列来预测用户下一个最可能会输入的词语;又例如有这样一个问题,需要让机器在阅读了一整篇文章后,回答若干关于这个文章的问题。这类问题和场景下都需要模型具有一定的记忆能力,能在获取新信息的同时,记住部分老的信息。

A) 最常用而有效的方式是使用一个递归神经网络模型(RNN \cite{4,5})。正如其名字描述的,递归神经网络在隐层结构上存在一个循环,即当前隐层的输入是上一个隐层的输出以及当前的输入2项一起。由于每个隐层的信息都能递归的输入到下一个隐层中,因此会具有一定的记忆能力。如图\ref{fig:RNN},我们将RNN按时间序列“打开”,可以看到前一时刻的隐层$S_{t-1}$和当前输入$X_t$会共同影响当前的隐层$S_{t}$。

\begin{figure}[h]
	\centering
	\includegraphics[width=0.8\linewidth]{"fig/RNN"}
	\caption{递归神经网络(RNN)示意图}
	\label{fig:RNN}
\end{figure}	

然而RNN存在的最大问题是“梯度消失和爆炸”问题\cite{6}。这是因为在神经网络进行反向传播(backpropagation)的时候,传播的梯度会是$w_{l,h}(t)$(递归网络的权重)的倍数;因此在递归层数较深的时候,梯度会消失掉(当$|w_{l,h}*y^{'}_{l}|<1$)或者爆炸(当$|w_{l,h}*y^{'}_{l}|>1$)。由于RNN存在“梯度消失和爆炸”问题,RNN的“记忆”只能是很短期的,并不具备长期的记忆。

B) 为了解决梯度消失和爆炸”问题,一种更为巧妙的递归网络结构LSTM(Long Short-Term Memory)cite{5}被设计了出来。在LSTM中,RNN中递归的隐层单元被一个存储单元(LSTMUnit)所替代,每个存储单元由一个输入门(InputGate),一个输出门(OutputGate)和一个长期的内部的通过遗忘门(ForgetGate)更新的内部状态(Cell)相关联,如图\ref{fig:LSTM}。

\begin{figure}[h]
	\centering
	\includegraphics[width=0.3\linewidth]{"fig/LSTM"}
	\caption{LSTM(Long Short-Term Memory)示意图}
	\label{fig:LSTM}
\end{figure}

内部状态Cell可以可以理解为模型存储的长期记忆:每进行一次递归迭代的时候,Cell会通过遗忘门遗忘掉部分记忆,同时通过输入门决定当前输入有多少有效信息是需要被记住的,从而得到新的记忆。最终的输出通过当前新的记忆得到,由输出们决定新的记忆中哪些是当前需要的。在LSTM的反向传播过程中,不同于RNN中梯度是一个连乘的形式(由于链式法则),可以转化成一个连加的形式,因此有效的避免了梯度的消失和爆炸,从而具备一定的长期记忆的能力。在基础的LSTM基础上,学者们提出了多种LSTM的变种,比如\cite{10}、GRU\cite{11}和 Clockwork RNN\cite{12},他们在计算性能上会有较大的差别,然而效果基本没太大差距\cite{13,14}。一个基本的LSTM更新公式如下:
\begin{eqnarray}
i_t &=& \sigma(W_{hi} * h_{t-1} + W_{xi} * x_t + b_i)
\\
f_t &=& \sigma(W_{hf} * h_{t-1} + W_{xf} * x_t + b_f)
\\
o_t &=& \sigma(W_{ho} * h_{t-1} + W_{xo} * x_to+ b_o)
\\
g_t &=& tanh(W_{hg} * h_{t-1} + W_{xg} * x_t + b_g)
\\
c_t &=& (f_t .* c_{t-1}) + (i_t .* g_t)
\\
h_t &=& o_t .* tanh(c_t)
\end{eqnarray}

C) 递归神经网络外主要能有效的处理“序列”相关的问题,因此被大量的用在NLP的问题中。除了RNN外,也有一些其他的模型有类似功能,例如神经图灵机(Neural Turing Machine, NTM\cite{7})和记忆神经网络(Memory Networks,MenNN\cite{8,9}),他们在不同场景下会比RNN“记住”更久远的信息,从而得到更好的效果。神经图灵机的主要思想是使用一个$M*N$的矩阵取存储一份长期记忆(这与LSTM是类似的,只是LSTM维护的是一个向量),该矩阵和一个神经网络共同进行学习和预测。记忆矩阵会通过选择性的读和写来进行迭代更新,同时由于每部分都是可微的,因此可以通过梯度下降法进行训练。NTM的基本工作原理如下图:

\begin{figure}[h]
	\centering
	\includegraphics[width=0.5\linewidth]{"fig/NTM"}
	\caption{Neural Turing Machine(NTM)示意图}
	\label{fig:NTM}
\end{figure}

记忆神经网络\cite{8}主要用在长期记忆的推断,网络会从一个长文本中自动的将重要的信息编码后记录下来。最后产出的模型能回答关于长文本的任何问题——根据问题从记忆中寻找相关内容,然后产生答案。一个经典的MemNN的预测过程由简单的4步组成:
\begin{itemize}
	\item[-] 将输入$x$编码成一个隐向量$I(x)$。
	\item[-] 更新记忆$m_i$,$m_i=G(m_i, I(x), m)$。即通过当前的隐向量,当前记忆,整体记忆,去更新记忆中的一块内容。
	\item[-] 通过当天的记忆内容和输入决定输出向量。$o = O(I(x),m)$
	\item[-] 最后将输出向量解析成最终的回答。$r = R(o)$
\end{itemize}
然而MemNN的一个问题在于并不能End-to-End的去学习,同时NTM和MemNN并没有关注输入的顺序信息。

3,在个性化搜索中,最为重要的是怎么去理解和认识一个淘宝的用户。除了用户的一些基本画像信息,我们拥有最为关键的、与其他平台不同的数据是用户在淘宝上的行为。由于用户在淘宝上的行为天然是一个长期的行为序列,因此很自然考虑使用RNN等序列模型取进行处理。一个最基础的模型结构如图\ref{fig:MULTI-LSTM}。

\begin{figure}[h]
	\centering
	\includegraphics[width=0.5\linewidth]{"fig/MULTI-LSTM"}
	\caption{淘宝序列匹配模型示意图}
	\label{fig:MULTI-LSTM}
\end{figure}

从图中我们可以看到网络主要由3大部分组成,分别是:1,首先得到用户的行为序列;2,将用户的行为序列经过一个带记忆的网络编码成隐向量H;3,将H通过多目标网络训练不同的目标。那么下面我们将从这三方面详细说明淘宝搜索中的序列匹配模型。

a)首先是用户序列部分。我们使用用户有过行为的商品序列作为用户的表示,每一个商品被embedding到一个128维的向量中,这个向量可以从word2vec的方法进行无监督学习得到,也可以从一个长期fine tune的深度神经网络得到。在获得商品表示的算法中,一个商品embedding前的编码主要包括商品ID、店铺、类目、价格信息。

\begin{figure}[h]
	\centering
	\includegraphics[width=0.9\linewidth]{"fig/Item_embedding"}
	\caption{商品embedding}
	\label{fig:Item_embedding}
\end{figure}

商品的embedding部分是在训练训练网络前提前训练好的,我们并没有将其放到训练序列的网络中,主要是因为ID特征十分稀疏,在一个LSTM的网络中,数据可能并不支持训练这么大规模的特征维度,从而影响模型的整体效果。然而这样的问题是,预训练得到的商品只包含了基本的商品特征,可能并不最适合当前的序列匹配网络。因此受transfer learning的启发,
向量经过embedding的商品向量并不是直接作为特征输入到LSTM或者MemNN中,而是根据商品的行为类型(点击、成交、收藏、加购)和来源经过不同的卷积核生成一个新的128维向量,然后输入到序列网络。这样既对用户的行为进行了区分,可以学习得到不同行为的重要性;同时对预训练得到的商品向量往新的目标上调整。

b)在得到用户的商品行为序列后,我们需要使用一个序列或者记忆模型,将序列编码成一个通用的用户状态H。这里我们对比了LSTM和End-to-end memory network\cite{9}。LSTM在上文中已经有过一些介绍,而一个End-to-end memory network与经典的memNN的区别在于它可以通过一个整体的网络去学习,基本的网络结构如图\ref{fig:end2end}。它首先将输入序列中的每一项同时映射成2个向量$m_i$和$c_i$,分别表示“输入记忆”和“输出记忆”。“输入记忆”决定序列中每一项的重要性$p_i$,$p_i$和和$c_i$相乘求和得到输出向量$o$。输出向量$o$和用户向量决定最终的答案$a$。LSTM使用一个记忆单元$Cell$去记忆历史信息;而End-to-end memory network正着重于将原始序列压缩,并自动挖掘序列中元素的重要性。我们使用两种方法在大量数据上进行了实验对比,从AUC来看LSTM会略优于End-to-end memory network,但是End-to-end memory network在计算速度上会远优于LSTM。
\begin{eqnarray}
c_i &=& \sigma(W_{c} * x_{i}
\\
m_i &=& \sigma(W_{m} * x_{i}
\\
p_i &=& Softmax(u^T m_i)
\\
o &=& \sum_{i}p_i c_i
\\
a &=& Softmax(W(o+u))
\end{eqnarray}

\begin{figure}[h]
	\centering
	\includegraphics[width=0.9\linewidth]{"fig/end2end"}
	\caption{End-to-end memory network}
	\label{fig:end2end}
\end{figure}

c)在得到的用户的隐向量后,进行匹配是容易的,只需要一个不用太深的DNN网络对用户向量H和商品向量放到一起进行预测即可。但是为了学到更加鲁棒的网络结构,我们使用multi-task的相关技术\cite{16}建立了多个辅助目标共同学习。因为multi-task learning不是本章重点,因此不再详细介绍,结果部分会有相关对比结果。


4,结果展示
	
\section{排序学习}
	@元涵,@凌运, @龙楚
\subsection{}

\section{展示学习}
	个性化短标题:@苏哲,@仁重 

\section{模型参数优化} 
	@公达

\section{AI4B实战} 

\begin{thebibliography}{99}
\addcontentsline{toc}{chapter}{\protect\numberline{}{\hspace{-1.5em}参考文献}}
\markboth{参考文献}{参考文献}
\bibitem{1} 淘宝搜索全链路有效行为量化模型(UBM\&UCM), http://www.atatech.org/articles/38550
\bibitem{2} User Browsing Model的实现与应用, http://www.atatech.org/articles/23111
\bibitem{3} 搜索个性化介绍, http://www.atatech.org/articles/48548
\bibitem{4} Mikolov, T., Karafi´at, M., Burget, L., Cernock`y, J., Khudanpur, S.: Recurrent neural network based language model. J. Interspeech. 1045–1048 (2010)
\bibitem{5}  Hochreiter, S., Schmidhuber J.: Long short-term memory. J. Neural Computation. 9(8), 1735–1780 (1997)
\bibitem{6} Learning Long-Term Dependencies with Gradient Descent is Difficult
\bibitem{7} Graves, A., Wayne, G., Danihelka, I.: Neural Turing Machine. arXiv preprint:1410.5401v2 (2014)
\bibitem{8} Weston, J., Chopra, S., Bordes, A.: Memory Networks. C. International Conference on Learning Representations. arXiv:1410.3916 (2015)
\bibitem{9} Sukhbaatar, S., Szlam, A., Weston, J., Fergus, R.: End-To-End Memory Networks. J. Advances in Neural Information Processing Systems. 28, 2440–2448 (2015)
\bibitem{10} Gers, Felix A and Schmidhuber, J: Recurrent Nets that Time and Count. J. in IJCNN 2000
\bibitem{11} Cho, Kyunghyun and Van Merrienboer, Bart and Gulcehre, Caglar and Bahdanau, Dzmitry and Bougares, Fethi and Schwenk, Holger and Bengio, Yoshuai: Learning Phrase Representations using RNN Encoder–Decoder for Statistical Machine Translation. arXiv preprint arXiv:1406.1078 (2014)
\bibitem{12} Koutnik, Jan and Greff, Klaus and Gomez, Faustino and Schmidhuber, Juergen: A Clockwork RNN. J. arXiv preprint arXiv:1402.3511 (2014)
\bibitem{13} Zaremba, Wojciech: An Empirical Exploration of Recurrent Network Architectures. in LMLR 2015
\bibitem{14} Greff, Klaus and Srivastava, Rupesh K and Koutnik, Jan and Steunebrink, Bas R and Schmidhuber, J: A search space odyssey. IEEE transactions on neural networks and learning systems (2016)
\end{thebibliography}

 
