
\chapter{个性化搜索背后的核心技术}
\thispagestyle{empty}

\setlength{\fboxrule}{0pt}\setlength{\fboxsep}{0cm}
\noindent\shadowbox{
\begin{tcolorbox}[arc=0mm,colback=lightblue,colframe=darkblue,title=学习目标与要求]
%\kai\textcolor{darkblue}{1.~~强化学习.} \\ 

\end{tcolorbox}}
\setlength{\fboxrule}{1pt}\setlength{\fboxsep}{4pt} 

进入正题之前首先谈一谈个性化方向后续需要关注的几个点,搜索个性化时至今日,已经成为
互联网网站的技术标配,虽然业界取得了一些成绩,但挑战仍然存在;

首先来看看为什么要作个性化,搜索中引入个性化的目的是什么
每天有 近30000个查询“连衣裙”的消费者,query + “user context” 的查询逻辑能够实现不同的消费群体看到不同商品投放结果,实现平台上人-货匹配在搜索流量上的个性化细分,比如说,“肥胖”的女性查询结果里面更多的展现宽松风格的商品,而消费能力高的消费者更多的展现高品位的大牌商品,从而达到流量投放效率的最大化;总而言之,目的是两个 : 
a). 提升流量匹配效率:具体表现在购物路径上的效果指标;
b). 改善宽泛query下得流量集中性,提升宽泛query下不同人群看到的展示商品不同,而带来成交商品和点击商品的丰富性;
总而言之:对于广大消费者,由于个性化能够细分搜索意图,拟合个体偏好,有助于更快捷找到需求;

弄清楚了搜索个性化的目的,下面想来澄清几个问题:

1. 个性化并不是定制化

Customization:实现的境界是 You are what you say you are? 或者说,平台按照系统理解的用户profile,并按照某种特定个性化规则去投放,即是,实现 you are what the system thinks you are. 对于针对用户的查询结果的信息展示是局限在explicit 的“feature”层面,比方说,按照购买力匹配规则,按照品牌偏好规则等等;定制化的好处的确能够在某种程度上带来强的个性化体验,但是带来的伤害也是不言而喻;
而Personalization是:根据用户行为所挖掘的偏好信息来进行展示商品的投放,即是,实现 you are what you click on and what you buy;对于针对用户的查询结果的商品展示是基于“内容和数据”层面;不去刻意的假设用户的行为是由于某个特定维度(人口统计类特征,偏好类特征,人群特征)造成的,消费者点击或成交行为的发生,是所有个性化信息的综合表现;为什么我在这里先提出这个问题,因为经常听到的很多关于,目前线上个性化效果不尽人意的反馈,在这里,也不去刻意回避我们自己的问题,个性化数据,模型的覆盖率,准确性和时效性等都需要进一步的优化和改进;然而,对于那些为了增强所谓的个性化体验而实行的规则式匹配逻辑,都是极其不科学的做法,对于消费者而言,他们需要的是找到一个符合他/她需求的商品,而个性化体验强弱与否并非是最终的目的,我相信的是,消费者不会因为我们预测到他的性别,购买力,偏好的品牌就做出点击或购买的决策,个性化是我们系统实现高效的【人-货】匹配效率的手段,并非是消费者的购物诉求;在这里也请从事个性化方面的运营,产品,甚至算法同学能时刻理解这点;

2. 不要陷入活跃/资深用户的悖论

正常的用户无论其活跃与否,都不会愿意浪费时间去填写所谓的友好的交互式表单来帮助系统去理解他们,从而得到更好的个性化体验;他们关注的是展示商品整体是否满足他们的需求,而不会去刻意的由于商品的某个维度匹配了他/她得某个偏好而做出最终的选择;这里列举一个曾经的产品设计,在搜索结果页,给出用户可以定制的个性化偏好交互界面,希望消费者能告诉我们他们的个性化profile,出发点是好的,结局大家懂的;

3. 个性化explore的重要性

随着个性化元素在搜索全链路的渗透,从query的个性化标注,海选的个性化召回,精排中的个性化排序因子,以及个性化rerank,个性化展示,使得最终呈现给用户的内容取决于系统底层根据用户历史行为所挖掘的个性化特征,人口统计学维度,兴趣点偏好维度,session级别实时特征,过度的”user specific historical behaviour driven“的个性化投放,会使得用户逐渐丧失对展示结果的新鲜感,并且视野变得越来越狭窄, 进而使得底层的用户数据模型丧失自我修复和自我扩展能力;因此一个完整的个性化体系,必须考虑explore机制的设计环节;

4. 个性化评估的方法论

要想推动个性化效果的正向迭代,首先需要建立起合理的效果评估体系;然而这仍然是一个很大的问题,学术界流行的准确率,召回率,F1值,AUC,RMSE,等都有很大的局限性,这一层面的评估,只能保证数据模型的正确性;而在实际工作中,这些指标上的不一定能保证线上效果的收益;因此我们需要第二层次的评估手段,来看个性化算法效果。实现个性化的投放效果,是系统层面的主动而为,而且并没有去引导消费者端在一次搜索看到展示结果后,做出选择。 因此在消费者不知情情况下,消费者的行为反馈可以用来作为个性化效果评估的一个手段。对于已经上线的个性化特征,需要部署相应的统计分析模块,在ABtest机制下,监控各个特征的覆盖率,以及覆盖流量下的点击率,转化率等;虽然无法直接统计到这些特征对于点击和转化带来的精确影响,但是通过追踪高权重user的体验 - 点击率,2跳率,转化率等,能够了解这些特征的影响趋势,及早发现问题;这里特别强调下,采用高权重user的行为数据来分析的原因是,高权重用户意味着是活跃用户,意味着行为丰富,而这类用户的个性化特征的表现会更有代表性;最后,我也来谈谈对于针对个性化效果的社会化评测的意见和想法,便于理解,就以用户购买力为例来讲讲,为了更好来把握该维度数据的有效性,经常利用的手段是社会化评测来给定一些查询下,看看展示结果里面展示商品的价格是否符合评测者的价格偏好,从表象上看,似乎没有问题,然而,这里面确有一个本质上的问题,我们限定了这些参与评测人得判断角度,只关注商品价格,并给出满意与否的结论,而在实际购物场景下,用户对于商品满意与否接受与否,并不是只限定在价格本身,因此这样的评测还是有一定的局限性;我个人的观点,还是去真实的模拟线上的判断环境,不去刻意要求消费者去关注某个固定的维度,只是给出site by site的结果,让用户判断哪边展示的商品更符合他的口味,当然,这不同site的展示结果的差异,背后只是某个维度的个性化带来的影响,这样去评测,才更加客观;总结一句话,就是众包评测的关键是,希望参与者能做出客观的反馈,不应该做任何主观性的引导;

5. 个性化体系对于系统和框架的影响

在搜索场景下实现个性化的效果,就需要去建模分析 【query-user-商品】三元组构成下得海量数据分析,数据是极端稀疏的,算法时间和空间的复杂性,都对于体统能很好的支持分布和并行的数据分析和建模能力提出了很高的要求;另外,用户偏好的时效性,也需要我们能够实现增量,实时计算能力,个性化的实施,使得传统引擎依赖的性能优化利器,cache机制无法施展手脚,因此对于引擎的创新性改造也提出了更高的要求;另外,个性化数据的挖掘都是存在不确定性的,如何来设计一套能够保证误差不会累积的算法体系,也就是说需要建立一套数据自我修复的实时反馈体系,来保证由消费者端实时获取的客观反馈数据参与到个性化投放环节来保证模型的自我修复能力快于数据误差的传播速度;这样才能保证数据产生的价值形成良性的循环,构成大数据生态体系;



个性化是一种解决“长尾需求”的方式,“长尾理论”说的是用户需求集中度越来越低,用户和用户之间不一样,
我们如何来区分这种不一样? 个性化搜索就是融合推荐元素,以实现:用户个体需求主导的 “pull” 式搜索加
平台以数据驱动的方式对用户进行“push”式相关信息推送;


综述性的东西,@三桐,@公达

\section{匹配学习}
	
\subsection{一阶人货匹配模型} 
	@公达 u2i,u2s,u2b 

\subsection{高阶人货匹配模型} 
	@公达 u2i2i,u2s2i,u2b2i 

\subsection{深度匹配模型} 

\subsection{序列匹配模型} 
	
\section{排序学习}
	@元涵,@凌运, @龙楚
\subsection{}

\section{展示学习}
	个性化短标题:@苏哲,@仁重 

\section{模型参数优化} 
	@公达

\section{AI4B实战} 

\begin{thebibliography}{99}
\addcontentsline{toc}{chapter}{\protect\numberline{}{\hspace{-1.5em}参考文献}}
\markboth{参考文献}{参考文献}
\bibitem{1} 淘宝搜索全链路有效行为量化模型(UBM\&UCM), http://www.atatech.org/articles/38550
\bibitem{2} User Browsing Model的实现与应用, http://www.atatech.org/articles/23111
\bibitem{3} 搜索个性化介绍, http://www.atatech.org/articles/48548
\end{thebibliography}

 
