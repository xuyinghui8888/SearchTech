
\chapter{个性化搜索背后的核心技术}
\thispagestyle{empty}

\setlength{\fboxrule}{0pt}\setlength{\fboxsep}{0cm}
\noindent\shadowbox{
\begin{tcolorbox}[arc=0mm,colback=lightblue,colframe=darkblue,title=学习目标与要求]
%\kai\textcolor{darkblue}{1.~~强化学习.} \\ 

\end{tcolorbox}}
\setlength{\fboxrule}{1pt}\setlength{\fboxsep}{4pt} 

综述性的东西,@三桐,@公达

\section{匹配学习}
	
\subsection{一阶人货匹配模型} 
	@公达 u2i,u2s,u2b 

\subsection{高阶人货匹配模型} 
	@公达 u2i2i,u2s2i,u2b2i 

\subsection{深度匹配模型} 

\subsection{序列匹配模型} 
	
\section{排序学习}
	@元涵,@凌运, @龙楚
\subsection{}

\section{展示学习}
	个性化短标题:@苏哲,@仁重 

\section{模型参数优化} 
	@公达

\section{AI4B实战} 

\begin{thebibliography}{99}
\addcontentsline{toc}{chapter}{\protect\numberline{}{\hspace{-1.5em}参考文献}}
\markboth{参考文献}{参考文献}
\bibitem{1} 淘宝搜索全链路有效行为量化模型(UBM\&UCM), http://www.atatech.org/articles/38550
\bibitem{2} User Browsing Model的实现与应用, http://www.atatech.org/articles/23111
\bibitem{3} 搜索个性化介绍, http://www.atatech.org/articles/48548
\end{thebibliography}

 
