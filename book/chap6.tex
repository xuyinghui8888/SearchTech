
\chapter{消费者权益智能分发核心技术}
\thispagestyle{empty}

\setlength{\fboxrule}{0pt}\setlength{\fboxsep}{0cm}
\noindent\shadowbox{
\begin{tcolorbox}[arc=0mm,colback=lightblue,colframe=darkblue,title=学习目标与要求]
%\kai\textcolor{darkblue}{1.~~带约束优化问题} \\ 

\end{tcolorbox}}
\setlength{\fboxrule}{1pt}\setlength{\fboxsep}{4pt} 


在我们的业务场景下,经常会遇到这样的问题,有限的资源下,在尽可能满足多宿主资源
诉求的前提下最大化资源的利用效率,比方说一下问题: 
\begin{description}
	\item 广告展示场景:
	\item 包裹分配:
	\item 消费者权益发放:
\end{description}





\section{购物券/红包发放技术} 
	@达卿,@云志



\begin{thebibliography}{99}
\addcontentsline{toc}{chapter}{\protect\numberline{}{\hspace{-1.5em}参考文献}}
\markboth{参考文献}{参考文献}
\bibitem{1} http://www.atatech.org/articles/66486, 双11搜索关键词红包:商家、用户与平台的三方共赢
\bibitem{2} 双11关键词红包:搜索链路新型互动性产品探索, http://www.atatech.org/articles/44778
\bibitem{3} 淘宝外卖智能补贴算法, http://www.atatech.org/articles/72599 
\bibitem{4} 大数据下线性最优化问题solver介绍, https://www.atatech.org/articles/69242?commentId=115358#comment-115358

\end{thebibliography}

 
