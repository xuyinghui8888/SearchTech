
\chapter{实时计算背后的核心技术}
\thispagestyle{empty}

\setlength{\fboxrule}{0pt}\setlength{\fboxsep}{0cm}
\noindent\shadowbox{
\begin{tcolorbox}[arc=0mm,colback=lightblue,colframe=darkblue,title=学习目标与要求]
%\kai\textcolor{darkblue}{1.~~强化学习.} \\ 

\end{tcolorbox}}
\setlength{\fboxrule}{1pt}\setlength{\fboxsep}{4pt} 


非i.i.d的数据分布,non-stationary环境,
很多传统的bounding理论是不成立的;

\section{在线矩阵分解}
	@达卿,@席奈

\section{在线LTR} 
	@凌运 

\section {在线深度学习} 
	@京五

\section{大规模WDL模型} 

\section{LR stacking on GBDT} 

\section{AI4B实战} 

\begin{thebibliography}{99}
\addcontentsline{toc}{chapter}{\protect\numberline{}{\hspace{-1.5em}参考文献}}
\markboth{参考文献}{参考文献}
\bibitem{1} 搜索双链路实时计算体系@双11实战,http://www.atatech.org/articles/44909
\bibitem{2} 基于在线矩阵分解的淘宝搜索实时个性化, http://www.atatech.org/articles/38646
\bibitem{3} BP如何运行, http://www.offconvex.org/2016/12/20/backprop/
\end{thebibliography}

 
