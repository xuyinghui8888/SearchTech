
\chapter{ 系统架构 } 
\thispagestyle{empty}

\setlength{\fboxrule}{0pt}\setlength{\fboxsep}{0cm}
\noindent\shadowbox{
\begin{tcolorbox}[arc=0mm,colback=lightblue,colframe=darkblue,title=学习目标与要求]
%\kai\textcolor{darkblue}{1.~~.} \\ 

\end{tcolorbox}}
\setlength{\fboxrule}{1pt}\setlength{\fboxsep}{4pt} 


\section{ 泛搜索框架架构图 } 

对于一个完整的泛搜索框架,是一个复杂系统,
1. 狭义层面的搜索引擎,涉及到离线的数据schema化,索引构建,召回和排序模块;


\begin{figure}[h]
\centering
\includegraphics[totalheight=3.0in]{fig/20210708121500.png}
\caption{20210708121500} \label{fig:gansamples}
\end{figure}


\begin{enumerate}
   \item 1. hippo:调度系统,用于分配机器
   \item 2. suez\_ops: 管控平台,用于启动引擎与分发任务,详细介绍见第十章
   \item 3. suez:用于执行suez_ops分发的任务
   \item 4. deploy express:用于分发包,索引,配置等数据
   \item 5. build service:用于处理文档和构建索引
   \item 6. swift:用于实时消息的传递
   \item 7. cm2:用于服务的发布,外界可以通过cm2定位qrs与searcher
\end{enumerate}

ha3引擎通常是由一组Qrs和searcher组件,ranker组件,summary组件组成的:
\begin{enumerate}
\item. Qrs的作用是:对输入的查询做解析和校验,通过后把查询转发给searcher,收集合并searcher返回的结果,最后对结果进行一些加工并返回给用户;
\item. searcher组件:基本对应的是文档的召回任务。
\item. ranker组件:打分和排序服务或者。
\item. summary组件:文档的摘要服务。
\end{enumerate}

三个数据流:
1. 查询流:
	\begin{enumerate}
    \item a. 当用户提交一个查询时,通常会经过sp业务层。sp层主要处理业务相关的逻辑,
       例如访问QP对查询进行扩展纠错,访问igraph获取相关的个性化信息等。
       经sp层处理后查询已经被处理成引擎所认识的查询串,
    \item. SP从cm2中拿一台qrs机器,把查询转发到QRS
    \item. Qrs根据内部的cm2信息,把查询转发到相应的searcher
  	\item. searcher根据查询,返回结果给qrs
  	\item. qrs 合并多个searcher的结果返回给sp,再由sp返回给用户
  	\end{enumerate}

2. 控制流
	这里的控制流指包含:a.服务的启停 b.索引和配置的分发与切换 
	\begin{enumerate}
    a. 用户在suez_ops页面上配置HA引擎包,cm2,索引等信息后,点击启动会向suez admin发一个执行目标
    b. suez_admin会对这个目标进行分解,产生一个carbon执行的目标。
    c. carbon根据目标向hippo申请机器并在这些机器上安装ha包和启动suez worker。
       HA的Qrs与searcher其实一种 suez worker的实现。
    d. 进程启后QRS与searcher就会加载配置与索引,
    e. 服务ready后suze admin会把这些服务发布到指定的cm2结点下。
    \end{enumerate}

3. 数据流
	searcher需要的数据通常索引目录或实时文档,这个索引通常由build service定期生成,实时文档则通过swift获取。
	suez_ops会定期向build service获取新产出索引并把新索引的目标发送给searcher。当searcher收到目标后会向deploy express发起下载索引标,下载完后会加载新的索引。
	build service是一个文档处理与索引构建服务。当dump中心产出一份新数据时,其就会触发一次全量build。dump通常会把全量数据写到hdfs文件上,把实时数据写入到swift上。
	build service构建索引分为3步:         
	1. process文档,文档的来源可以是文件或swift上的消息,processor处理后会把处理结果写中到swift中
    2. build索引,builder从swift获取已经处理的文档并对这些文档构建小索引
    3. merge索引,merger把builder产出的小索引合成一个大的索引

build service除了构建全量索引后,还会定期的构建增量索引。searcher的实时数据则直接读取processor写到swift上数据:


	\begin{figure}[h]
	\centering
	\includegraphics[totalheight=3.0in]{fig/20210708122651.png}
	\caption{20210708122651} \label{fig:gansamples}
	\end{figure}


	Suez Turing 框架介绍


	\begin{figure}[h]
	\centering
	\includegraphics[totalheight=3.0in]{fig/20210708154912.png}
	\caption{20210708154912} \label{fig:gansamples}
	\end{figure}


	suez turing 中把suez的worker抽象成了一个local的执行引擎,把检索流程抽象成图的执行,其中图中节点是各种算子,边是数据,远程worker的调用抽象成远程算子图,目前suez turing的worker只支持tensorflow的图执行,未来可能会加入更多的图执行引擎;各个应用整体流程用一张全图来描述,在suez ops那边图进行切分,然后分配子图到各个worker进行加载,其带来的好处是,采用算子级别的复用,减少相似算子的重复开发,通过图画的架构,达到流程的灵活定制;算子执行的并行化,异步化可以降低latency,统一的插件机制为算法开发提供统一的接口;codegen等技术的应用可让更多的业务受益;小业务的全图local执行可减少服务的调用;

	

	\begin{figure}[h]
	\centering
	\includegraphics[totalheight=3.0in]{fig/20210708155805.png}
	\caption{20210708155805} \label{fig:gansamples}
	\end{figure}


	suze turing的服务接口

	采用arpc的rpc框架,其pb的定义接口如下,主要传输的数据是tensor



\begin{thebibliography}{99}
\addcontentsline{toc}{chapter}{\protect\numberline{}{\hspace{-1.5em}参考文献}}
\markboth{参考文献}{参考文献}
\bibitem{1} C. Burges, T. Shaked, etc.., Learning to rank 
using gradient descent. In Proceedings of the 22nd international 
conference on machine learning, ACM
\end{thebibliography}

 
