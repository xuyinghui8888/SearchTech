\documentclass[9pt]{article}
%\usepackage{amssymb,amsmath,epsfig,graphicx,CJK,setspace,multirow}
\usepackage{amssymb,amsmath,epsfig,graphicx,color}
\setcounter{page}{1}
\sloppy     % better line breaks

\title{DNN Back Propagation}

\begin{document}
\large{DNN Back Propagation}

\begin{itemize}
\item First of all, let's define a few functions and compute their derivatives.
\begin{itemize}
\item Sigmoid function: $\sigma(x)=\frac{1}{1+e^{-x}} \in (0,1)$.
\[\sigma' = \sigma (1 - \sigma) \]


\end{itemize}
\item Output layer (layer N)

Assume output unit $q$, hidden layer unit $m$.
\begin{itemize}
\item $I_q$: summed input for unit $q$. 
\item $O_q$: softmax output of layer N, unit $q$
\item $W^N$: weight matrix from layer N-1 to layer N.
\item $B^N$: bias for layer N.
\end{itemize}
For simplicity, the layer superscript is dropped whenever there is no ambiguity.
As you will see, we will use a different index letter for the nodes at
each different layer. So it's no ambiguous at all to tell which layer a node
is located. Hence superscript is often not necessary. $soft(.)$ is the softmax function.

\begin{center}
\begin{tabular}{|l|l|} \hline
layer & node \\ \hline \hline
$N$ & $q$ \\ \hline
$N-1$ & $m$ \\ \hline
$N-2$ & $i$ \\ \hline
$N-3$ & $j$ \\ \hline
\end{tabular}
\end{center}

At each time frame $t$ is ($t$ is omitted for simplicity whenever possible):
\begin{eqnarray*}
I_q&  = &(\sum_m w_{mq} O_m) + b_q \\
O_q & = & soft(I_q) = \frac{e^{I_q}}{\sum_k e^{I_k}} \\
\end{eqnarray*}

Notice $O_m$ denotes the output of unit $m$, as we do for the output
of unit $q$. However, keep in mind that $q$ is a node at layer $N$, while
$m$ is a node at layer $N-1$. 

In back propagation algorithm for conventional DNN, the weight 
matrix is updated via stochastic gradient descent:
\begin{eqnarray*}
\hat{w}_{mq} & = & w_{mq} - \alpha \frac{\partial F}{\partial w_{mq}} \\
\hat{b}_{q} & = & b_q - \alpha \frac{\partial F}{\partial b_q}
\end{eqnarray*}
where $F$ is the objective function. 
With SGD, we compute the error signal and update the parameters per frame of
input. Therefore the objective function does not need to be summed across
all training data, but just one frame of data. Thus
minimizing the cross-entropy (CE) objective function for each time frame follows:
\begin{eqnarray*}
R =  \mbox{reference labels} & = & \left[0 0 ... 1 0 ... 0 \right] \\
O = \mbox{output labels} & = & \left[ O_1 O_2 ... O_{q^*} ... O_N \right] \\
\mbox{min}\;F(O) = H(R,O) & \triangleq & -\sum_i R_i \log {O_i} \\
& = & 0 + 0 + ... - \log O_{q^*} + 0 + 0 \\
& = & - \log O_{q^*} 
\end{eqnarray*}
where $q^*$ is the reference senone label at time $t$.

\begin{eqnarray*}
\frac{\partial F}{\partial w_{mq}} & = & \frac{\partial F}{\partial O_{q^*}} 
\frac{\partial O_{q^*}}{\partial I_q} \frac{\partial I_q}{\partial w_{mq}} \\
 & = & - \frac{1}{O_{q^*}} {\color{red} \frac{\partial O_{q^*}}{\partial I_q}} O_m \\
%& = & - \frac{1}{O_{q^*}} O_{q^*} (\delta_{qq^*} - O_q) O_m = -(\delta_{qq^*} - O_q) O_m \\
\end{eqnarray*}

For node $q^*$, its output is a softmax function. To compute the partial
derivative of $O_{q^*}$  w.r.t. $I_q$, let's do some homework first.
For simplicity, only in this paragraph, let's use q to also denote the input 
to node $q$ (i.e., $q=I_q$).

%\frac{\partial O_{q^*}}{\partial I_q}
\begin{eqnarray*}
s(q^*) = O_{q^*} & = & \frac{e^{q^*}}{\sum_q e^q} \\
\frac{\partial O_{q^*}}{\partial \color{red}I_{q^*}} = s'(q^*) & = & \frac{e^{q^*}}{\sum_q e^q} + e^{q^*}(-1){({\sum_q e^q})}^{-2} e^{q^*} \\
 & = & s(q^*) - s(q^*)s(q^*) = s(q^*) \left( 1-s(q^*) \right) \\
 & = & O_{q^*} (1-O_{q^*}) \\
\frac{\partial O_q}{\partial \color{red}I_q} = s'(q)  & = &  e^{q^*}(-1){({\sum_q e^q})}^{-2} e^q \\
  & = & -s(q)s(q^*) = -O_q O_{q^*} = O_{q^*} ( 0 - O_q) \\
\end{eqnarray*}
We can combine the above two formulas into one, using the Kronecker delta:
\[ \frac{\partial O_{q^*}}{\partial I_q} = O_{q^*} (\delta_{qq^*} - O_q) \]

Go back to the derivative of the objective function w.r.t. $w_{mq}$:
\begin{eqnarray*}
\frac{\partial F}{\partial w_{mq}} & = & - \frac{1}{O_{q^*}} O_{q^*} (\delta_{qq^*} - O_q) O_m \\
& = &  -(\delta_{qq^*} - O_q) O_m \\
\end{eqnarray*}


%http://www.emerson.emory.edu/services/latex/latex_119.html
If we define 
the error signal $\xi_q $ of output unit $q$ to be the negative derivative of 
the objective function w.r.t. its input, then
\begin{eqnarray*}
\xi_q & \triangleq & - \frac{\partial F}{\partial I_q} = \delta_{qq^*} - O_q = \; \\
\hat{w}_{mq} & = &  w_{mq} + \alpha \xi_q  O_m \\
\hat{b}_{q} & = & b_q + \alpha \xi_q
\end{eqnarray*}
where $\xi$ is the error between the reference (0,1) and the output.

\item Hidden layer N-1

Now consider hidden unit $m$ at layer N-1, and unit $i$ at layer N-2.
The error signal of hidden unit $m$ is the negative partial derivative
of the objective function w.r.t. its input:
\begin{eqnarray*}
\xi_m & \triangleq & - \frac{\partial F}{\partial I_m} =  - \frac{\partial F}{\partial O_m} \frac{\partial O_m}{\partial I_m} \\
 & = & \left( \sum_q \frac{-\partial F}{\partial I_q} w_{mq} \right) \frac{\partial O_m}{\partial I_m} \\
% & = & \{\sum_q \xi_q w_{mq}\} (1- O_m) O_m \\
%\hat{w}_{im} & = & w_{im} + \alpha \xi_m O_i \\
%\hat{b}_{m} & = & b_m + \alpha \xi_m \\
\end{eqnarray*}

The output of unit $m$ is a sigmoid function, and hence its derivative is : 

\begin{eqnarray*}
O_m & = & \sigma(I_m) = \frac{1}{1+e^{-I_m}} = {(1+e^{-I_m})}^{-1}  \\
\frac{\partial O_m}{\partial I_m} & = & -{(1+e^{-I_m})}^{-2} (-1) e^{-I_m}  \\
& = & {(1+e^{-I_m})}^{-1}  \frac{e^{-I_m}}{1+e^{-I_m}} \\
& = & O_m (1 - O_m) \\
\end{eqnarray*}

Hence,
\begin{eqnarray*}
\xi_m & = & (\sum_q \xi_q w_{mq}) (1 - O_m) O_m \\
\hat{w}_{im} & = & w_{im} + \alpha \xi_m O_i \\
\hat{b}_{m} & = & b_m + \alpha \xi_m \\
\end{eqnarray*}


\item Hidden layer N-2

Similarly, for hidden unit $i$ at layer N-2, and unit $j$ at layer N-3,
the error signal of unit $i$ is
\begin{eqnarray*}
\xi_i & \triangleq & - \frac{\partial F}{\partial I_i} =  - \frac{\partial F}{\partial O_i} \frac{\partial O_i}{\partial I_i} \\
 & = & \{\sum_m \xi_m w_{im}\} (1- O_i) O_i \\
\hat{w}_{ji} & = & w_{ji} + \alpha \xi_i O_j \\
\hat{b}_{i} & = & b_i + \alpha \xi_i \\
\end{eqnarray*}

If $j$ is the input layer (i.e. Layer 1), then $O_j = I_j$.
\end{itemize}
\end{document}


