
\chapter{ 基础排序算法技术 }
\thispagestyle{empty}

\setlength{\fboxrule}{0pt}\setlength{\fboxsep}{0cm}
\noindent\shadowbox{
\begin{tcolorbox}[arc=0mm,colback=lightblue,colframe=darkblue,title=学习目标与要求]
%\kai\textcolor{darkblue}{1.~~强化学习.} \\ 

\end{tcolorbox}}
\setlength{\fboxrule}{1pt}\setlength{\fboxsep}{4pt} 

\section{商品销量预测与人气分模型 }

\section{详情页满意度模型} 
@仁重

\section{用户浏览模型\&用户点击满意度模型}  

\section{网络效应分}


\begin{thebibliography}{99}
\addcontentsline{toc}{chapter}{\protect\numberline{}{\hspace{-1.5em}参考文献}}
\markboth{参考文献}{参考文献}
\bibitem{1} C. Burges, T. Shaked, etc.., Learning to rank 
using gradient descent. In Proceedings of the 22nd international 
conference on machine learning, ACM
\bibitem{2} 流量个性化v.s商业化 - 双11珠峰项目中控算法, http://www.atatech.org/articles/67132
\bibitem{3} 确定性保证下流量分配在线全局优化策略, http://www.atatech.org/articles/55983
\bibitem{4} 搜索流量确定性项目总结, http://www.atatech.org/articles/59651
\bibitem{5} 网络效应分介绍, https://www.atatech.org/articles/52962
\bibitem{6} Unbiased Learning-to-Rank with Biased Feedback, http://weibo.com/ttarticle/p/show?id=2309404077533346815648
\end{thebibliography}

 
