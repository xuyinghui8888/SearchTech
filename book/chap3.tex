
\chapter{搜索词背后的技术}
\thispagestyle{empty}

\setlength{\fboxrule}{0pt}\setlength{\fboxsep}{0cm}
\noindent\shadowbox{
\begin{tcolorbox}[arc=0mm,colback=lightblue,colframe=darkblue,title=学习目标与要求]
%\kai\textcolor{darkblue}{1.~~强化学习.} \\ 

\end{tcolorbox}}
\setlength{\fboxrule}{1pt}\setlength{\fboxsep}{4pt} 
淘宝的平台上有数十亿的商品,消费者在平台上想要快速找到自己想买的商品,只能在淘宝搜索输入查询词,也就是我们通常说的query,来表达购物的需求。如果能够理解用户Query背后的购物意图,就能够帮助搜索引擎自动将符合用户意图的商品返回给用户,提升结果的准确率,从而提高用户在平台上的购物满意度和体验。

\section{底纹推荐技术} 
	底纹推荐技术, 实际上是实现一个从用户到query的映射模型:$user \to query$

\section{查询词改写扩展技术}
	查询词改写扩展技术, 转换为技术语言,是一个完成从原始query到新query的映射模型:$query \to query^*$
\section{查询词意图预测技术}
查询词意图预测技术, 意图可以类目:$query \to category$


\section{查询词图像化映射技术} 

\section{AI4B实战@查询词图像化映射技术} 


\begin{thebibliography}{99}
\addcontentsline{toc}{chapter}{\protect\numberline{}{\hspace{-1.5em}参考文献}}
\markboth{参考文献}{参考文献}
\bibitem{1} C. Burges, T. Shaked, etc.., Learning to rank 
using gradient descent. In Proceedings of the 22nd international 
conference on machine learning, ACM
\end{thebibliography}

 
