
\chapter{序言}
\thispagestyle{empty}

\setlength{\fboxrule}{0pt}\setlength{\fboxsep}{0cm}
\noindent\shadowbox{
\begin{tcolorbox}[arc=0mm,colback=lightblue,colframe=darkblue,title=学习目标与要求]
%kai\textcolor{darkblue}{1.~~对抗学习.} \\ 

\end{tcolorbox}}
\setlength{\fboxrule}{1pt}\setlength{\fboxsep}{4pt}

淘宝搜索作为平台的一个重要联系买家和卖家的产品形态, 
由于其以下的特有属性,使其成为大数据智能化应用的最佳场景;
1. 海量消费者与平台的互动行为
2. 海量商家在平台进行的商业活动行为
3. 海量的商品
算法及模型在搜索和推荐系统领域占据统治地位之前,具有领域知识的专业运营和
产品往往充当信息展示规则的缔造者,根据主观的判断和对市场的敏锐度来制定
查询词背后的商品展示逻辑。“人工规则”的好处是容易理解和操控,坏处则不言而喻,
随着平台规模的增大,简单规则无法精细的表达人货匹配的效率,并且容易被一些
不良商家利用规则来扰乱市场秩序;实际上,早期的搜索和推荐系统也会运用一些
基本的算法逻辑来保证信息匹配的正确性和人货匹配的公平性,基于传统搜索
引擎技术的相关性模型,保证用户查询词语商品标题的有效匹配;基于商品成交
与否的销售人气指数模型,保证有助于被消费者接受的商品得到更多的展示机会;
另外还有一个就是系统为了保证让更多商家有机会得到展现,设置的按照虚拟下架
周期为参考的轮播因子,即将下架的商品会得到相对较高的展示机会。
$$
	score(item)=1-\frac{ItemOffshelfTime-QueryTime}{secondsOfTwoweek}\times(\frac{docFound}{delta})
$$
第一代搜索算法技术的一些弊端让它们已经不能适应现代


作为一个公平的市场调节员,调整供需平衡,为卖家引导潜在的消费群体,以提升其ROI(return on investment),为用户提供满足其需求(user intent)的商品;商业流量下的搜索自然带有其特有的技术特点:

淘宝搜索算法技术演进之路可以分为四个阶段,如图所示: 


\begin{thebibliography}{99}
\addcontentsline{toc}{chapter}{\protect\numberline{}{\hspace{-1.5em}参考文献}}
\markboth{参考文献}{参考文献}
\bibitem{1} Bilinear+LinUcb的个性化主题推荐, http://www.atatech.org/articles/67847
\bibitem{2} 流量个性化v.s商业化 - 双11珠峰项目中控算法, http://www.atatech.org/articles/67242
\bibitem{3} 依托搜索技术的个性化平台之路, http://www.atatech.org/articles/13748
\bibitem{4} 用户意图预估之实时意图篇, http://www.atatech.org/article/detail/12636/152
\bibitem{5} 知人知面需知心——论人工智能技术在推荐系统中的应用,http://geek.csdn.net/news/detail/112318


\bibitem{3} Liya Tolstikhin,  AdaGAN: boosting generative models


\end{thebibliography}

 
